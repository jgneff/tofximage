% ======================================================================
% chart.tex - creates bar charts of benchmark results
% Copyright (C) 2019 John Neffenger
%
% This program is free software: you can redistribute it and/or modify
% it under the terms of the GNU General Public License as published by
% the Free Software Foundation, either version 3 of the License, or
% (at your option) any later version.
%
% This program is distributed in the hope that it will be useful,
% but WITHOUT ANY WARRANTY; without even the implied warranty of
% MERCHANTABILITY or FITNESS FOR A PARTICULAR PURPOSE.  See the
% GNU General Public License for more details.
%
% You should have received a copy of the GNU General Public License
% along with this program.  If not, see <http://www.gnu.org/licenses/>.
% ======================================================================
\documentclass[12pt,tikz,border=3mm]{standalone}
\usepackage[T1]{fontenc}
\usepackage[utf8]{inputenc}
\usepackage{lmodern}
\usepackage{libertine}
\usepackage{microtype}
\usepackage{pgfplots}
\pgfplotsset{compat=newest}

\providecommand{\pdftitle}{Converting AWT Images to JavaFX Images}
\providecommand{\cpu}{Processor information}
\providecommand{\data}{dell.dat}
\pdfinfo{
    /Title (\pdftitle)
    /Subject (Benchmark results using the Java Microbenchmark Harness)
    /Author (John Neffenger)
    /Keywords (AWT, JavaFX, toFXImage, JMH)
}

\begin{document}
\begin{tikzpicture}[font=\sffamily]
\begin{axis}[
    title=\pdftitle\\\small{\cpu},
    title style={align=center},
    width=148mm,
    height=105mm,
    xbar,
    xmin=0,
    xmajorgrids,
    xlabel=Throughput (frames per second),
    x tick label style={/pgf/number format/assume math mode=true},
    symbolic y coords={swingFXUtils,streamParallel,streamOrdered,%
        nioGetPut,nioGetOnly,nioDrawPut,getSetPre,getSet,forLoops,%
        drawSetPre,drawSet,drawPreSetPre,drawPreSet},
    ytick=data,
]
\addplot[draw=black,fill=lightgray,error bars/.cd,x dir=both,x explicit]
table[x=Score,y=Benchmark,x error=Error,y error=Error]{\data};
\end{axis}
\end{tikzpicture}
\end{document}
